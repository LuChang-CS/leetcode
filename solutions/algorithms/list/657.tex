\tocless\section{657. Judge Route Circle}
\label{algo:657}

\subsection*{Difficulty}
Easy

\subsection*{Tags}
Entertainment

\subsection*{Description}
Initially, there is a Robot at position \inlinecode{(0, 0)}. Given a sequence of its moves, judge if this robot makes a circle, which means it moves back to the original place.

The move sequence is represented by a string. And each move is represent by a character. The valid robot moves are \inlinecode{R} (Right), \inlinecode{L} (Left), \inlinecode{U} (Up) and \inlinecode{D} (down). The output should be true or false representing whether the robot makes a circle.

\begin{example}
\begin{multilinecode}
Input: "UD"
Output: true
\end{multilinecode}
\end{example}

\begin{example}
\begin{multilinecode}
Input: "LL"
Output: false
\end{multilinecode}
\end{example}

\subsection*{Analysis}
I don't know how to classify this problem, so I just tag it as an entertainment.

Obviously, each point has a coordinate \inlinecode{(x, y)}. \inlinecode{R} means \inlinecode{x + 1}, \inlinecode{L} means \inlinecode{x - 1}, \ininecode{U} means \inlinecode{y + 1} and \inlinecode{D} means \inlinecode{y - 1}. Therefore, the robot's moving back to the original place means the coordinate is still \inlinecode{(0, 0)} after all moves, assuming original coordinate is \inlinecode{(0, 0)}.

\begin{itemize}
\item Time Complexity: $\mathcal{O}(n)$
\item Space Complexity: $\mathcal{O}(1)$
\end{itemize}

\subsection*{Solution}
\subsubsection*{C}
\begin{minted}[framesep=2mm,
baselinestretch=1.2,
bgcolor=codebackground,
fontsize=\footnotesize,
breaklines,
linenos]{c}
bool judgeCircle(char* moves) {
    int x = 0, y = 0;
    char move;
    for (int i = 0; (move = moves[i]) != '\0'; ++i) {
        switch (move) {
            case 'U':
            y += 1;
            break;
            case 'D':
            y -= 1;
            break;
            case 'R':
            x += 1;
            break;
            case 'L':
            x -= 1;
            break;
            default:
            break;
        }
    }
    return (x == 0 && y == 0);
}
\end{minted}

\newpage

