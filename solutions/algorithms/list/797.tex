\tocless\section{797. All Paths From Source to Target}
\label{algo:797}

\subsection*{Difficulty}
Medium

\subsection*{Tags}
Graph

\subsection*{Description}
Given a directed, acyclic graph of \inlinecode{N} nodes.  Find all possible paths from node \inlinecode{0} to node \inlinecode{N - 1}, and return them in any order.

The graph is given as follows:  the nodes are \inlinecode{0, 1, ..., graph.length - 1}.  \inlinecode{graph[i]} is a list of all nodes \inlinecode{j} for which the edge \inlinecode{(i, j)} exists.

\begin{example}
\begin{multilinecode}
Input: [[1, 2], [3], [3], []]
Output: [[0, 1, 3], [0, 2, 3]]

Explanation: The graph looks like this:
0--->1
|    |
v    v
2--->3
There are two paths: 0 -> 1 -> 3 and 0 -> 2 -> 3.
\end{multilinecode}
\end{example}

\textbf{Note}
\begin{itemize}
    \item The number of nodes in the graph will be in the range \inlinecode{[2, 15]}.
    \item You can print different paths in any order, but you should keep the order of nodes inside one path.
\end{itemize}

\subsection*{Analysis}
This is a classic DFS or BFS algorithm. We can easily use recursive method to generate paths.

Let me explain the example first. We first assume that the node number in the graph is \inlinecode{n}, and the node is \inlinecode{0, 1, ..., n - 1}. The input is \inlinecode{[[1, 2], [3], [3], []]}, which means node \inlinecode{0} is connected to \inlinecode{1} and \inlinecode{2} and so forth. In a recursive manner, we first fetch \inlinecode{1} and find \inlinecode{1} is connected to \inlinecode{3}, then these two nodes forms a path. And we next fetch \inlinecode{2}, and find \inlinecode{2} is also connected to \inlinecode{3}, hence, there are two paths. In this manner, we call is Depth-First-Search (DFS), and DFS can be intuitively implemented in a recursive manner.

As for Breadth-First-Search (BFS), BFS is more appropriate to be implemented in a loop manner with a queue. It fetch all nodes in a layer to this queue, dequeue each node, and put successive nodes in queue. In this example, we first fetch \inlinecode{1} and \inlinecode{2}. Then we fetch \inlinecode{3} and \inlinecode{3} as successive nodes of \inlinecode{1} and \inlinecode{2}. Finally we also find there are two paths.

In our solution, we use DFS in a non-recursive manner, just to try more method. We use stack to replace recursion. Similar to BFS, It first fetch all nodes in a layer to the stack, pop from stack and push its successive nodes.
\begin{itemize}
    \item Time complexity: $\mathcal{O}(n^2)$, when the graph is a complete graph, because node \inlinecode{i} have \inlinecode{n - i} successive nodes.
    \item Space complexity: $\mathcal{O}(n^2)$ , also when it is a complete graph.
\end{itemize}

\subsection*{Solution}
\subsubsection{C}
\begin{minted}[framesep=2mm,
baselinestretch=1.2,
bgcolor=codebackground,
fontsize=\footnotesize,
breaklines,
linenos]{c}
// use a linked list to store path temporarily
struct Path {
    int *path;
    int path_len;
    struct Path *next;
};
typedef struct Path Path;

int** allPathsSourceTarget(int** graph, int graphRowSize, int *graphColSizes, int** columnSizes, int* returnSize) {
    Path *paths_list = (Path *)malloc(sizeof(Path));
    paths_list->next = NULL;
    Path *paths_list_tail = paths_list;
    int path_number = 0;
    
    // BFS in a stack manner
    int node_stack[225], top = 0;
    node_stack[0] = 0;
    // this stack is to store the path length in each layer
    // the successive nodes of one node form a layer
    int path_len_stack[225], len_top = 0;
    path_len_stack[0] = 1;
    
    // a path
    int *current_path = (int *)malloc(15 * sizeof(int));
    int current_path_len;
    while (top >= 0) {
        // fetch next node
        int current_node = node_stack[top--];
        // fetch the coresponding path length for this node
        current_path_len = path_len_stack[len_top--];
        // add this node to current path
        current_path[current_path_len - 1] = current_node;
        if (current_node == graphRowSize - 1) {  // if path exists
            // add path to linked list
            Path *p = (Path *)malloc(sizeof(Path));
            p->path = current_path;
            p->path_len = current_path_len;
            p->next = NULL;
            paths_list_tail->next = p;
            paths_list_tail = p;
            
            // get a new path based on current path
            current_path = (int *)malloc(15 * sizeof(int));
            memcpy(current_path, p->path, (current_path_len - 1) * sizeof(int));
            ++path_number;
        } else {
            int *current_row = graph[current_node];
            int current_row_size = graphColSizes[current_node];
            // add successive nodes and path lengths to the stack
            for (int i = 0; i < current_row_size; ++i) {
                node_stack[++top] = current_row[i];
                path_len_stack[++len_top] = current_path_len + 1;
            }
            // move to next layer
            if (current_row_size > 0) ++current_path_len;
        }
    }
    
    *returnSize = path_number;
    int **paths = (int **)malloc(path_number * (sizeof(int *)));
    *columnSizes = (int *)malloc(path_number * (sizeof(int *)));
    Path *p = paths_list->next;
    // transform linked list to 2d array
    for (int i = 0; p != NULL; ++i) {
        paths[i] = p->path;
        (*columnSizes)[i] = p->path_len;
        p = p->next;
    }
    return paths;
}

\end{minted}

\newpage

