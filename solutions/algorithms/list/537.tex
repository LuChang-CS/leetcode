\tocless\section{537. Complex Number Multiplication}
\label{algo:537}

\subsection*{Difficulty}
Easy

\subsection*{Tags}
Math, String

\subsection*{Description}
Given two strings representing two \href{https://en.wikipedia.org/wiki/Complex_number}{complex numbers}.

You need to return a string representing their multiplication. Note $i^2 = -1$ according to the definition.

\begin{example}
\begin{multilinecode}
Input: "1+1i", "1+1i"
Output: "0+2i"
Explanation: (1 + i) * (1 + i) = 1 + i2 + 2 * i = 2i, and you need convert it to the form of 0+2i.
\end{multilinecode}
\end{example}

\begin{example}
\begin{multilinecode}
Input: "1+-1i", "1+-1i"
Output: "0+-2i"
Explanation: (1 - i) * (1 - i) = 1 + i2 - 2 * i = -2i, and you need convert it to the form of 0+-2i.
\end{multilinecode}
\end{example}

\textbf{Note}
\begin{itemize}
    \item The input strings will not have extra blank.
    \item The input strings will be given in the form of \textbf{a+bi}, where the integer a and b will both belong to the range of [-100, 100]. And \textbf{the output should be also in this form}.
\end{itemize}

\subsection*{Analysis}
This is an easy complex multiplication problem with many details. First you should parse the string to a complex number. Secondly you should compute the multiplication, and finally you need to transform it to a string in the required form. To parse the complex number, we can easily think up that the part before \inlinecode{+} is the real and the part between \inlinecode{+} and \inlinecode{i} is the image. And to transform the complex to string, we only need to call \inlinecode{sprintf}. However, if we first find the position of \inlinecode{+} and \inlinecode{i} and call \inlinecode{atoi} to transform string to integers, we will get extra iterations. Therefore, we can implement the transform function by ourselves.

We assume the max length of complex strings is $n$, then
\begin{itemize}
	\item Time Complexity: $\mathcal{O}(n)$
	\item Space Complexity: $\mathcal{O}(1)$
\end{itemize}

\subsection*{Solution}
\subsubsection*{C}
\begin{minted}[framesep=2mm,
baselinestretch=1.2,
bgcolor=codebackground,
fontsize=\footnotesize,
linenos]{c}
struct Complex {
    int real;
    int image;
};
typedef struct Complex Complex;

int _itoa(int a, char *s) {
    int i = 0, start = 0;
    // zero should also be printed
    if (a == 0) {
        s[0] = '0';
        s[1] = '\0';
        return 1;
    }
    if (a < 0) {
        a = -a;
        s[0] = '-';
        start = i = 1;
    }
    while (a > 0) {
        s[i++] = (char)((a % 10) + '0');
        a /= 10;
    }
    // reverse number because steps above fetch digit from lower end to higher.
    for (int k = start; k < i / 2 + start; ++k) {
        char c = s[k];
        int pos = i - k - (1 - start);
        s[k] = s[pos];
        s[pos] = c;
    }
    s[i] = '\0';
    return i;
}

void parse_complex(Complex *complex, char *s) {
    char c;
    int val = 0;
    int sign = 1;
    for (int i = 0; (c = s[i]) != '\0'; ++i) {
        switch (c) {
        case '+':
            // val now is the real part
            complex->real = sign == 1 ? val : -val;
            // reset val and sign after real part to calculate image part
            val = 0;
            sign = 1;
            break;
        case 'i':
            // val now is the image part
            complex->image = sign == 1 ? val : -val;
            case '-':
            sign = -1;
            break;
        default:
            val *= 10;
            val += c - '0';
        }
    }
}

char* complexNumberMultiply(char* a, char* b) {
    Complex c1, c2;
    parse_complex(&c1, a);
    parse_complex(&c2, b);
    int real = c1.real * c2.real - c1.image * c2.image;
    int image = c1.real * c2.image + c1.image * c2.real;
    char *result = (char *)malloc(20 * sizeof(char));
    int pos = _itoa(real, result);
    result[pos++] = '+';
    pos += _itoa(image, result + pos);
    result[pos++] = 'i';
    result[pos] = '\0';
    return result;
}
\end{minted}

\newpage
