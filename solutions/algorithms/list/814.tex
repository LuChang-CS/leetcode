\tocless\section{814. Binary Tree Pruning}
\label{algo:814}

\subsection*{Difficulty}
Medium

\subsection*{Tags}
Binary Tree, Recursive Algorithm

\subsection*{Description}
We are given the head node \inlinecode{root} of a binary tree, where additionally every node's value is either a 0 or a 1.

Return the same tree where every subtree (of the given tree) not containing a 1 has been removed.

(Recall that the subtree of a node X is X, plus every node that is a descendant of X.)
\begin{example}
\begin{multilinecode}
Input: [1, null, 0, 0, 1]
Output: [1, null, 0, null, 1]

Explanation:
Only the red nodes satisfy the property "every subtree not containing a 1".
The diagram on the right represents the answer.
\end{multilinecode}
\includegraphics*[width=14cm]{figs/algo_814_1}
\end{example}

\begin{example}
\begin{multilinecode}
Input: [1, 0, 1, 0, 0, 0, 1]
Output: [1, null, 1, null, 1]
\end{multilinecode}
\includegraphics*[width=14cm]{figs/algo_814_2}
\end{example}

\begin{example}
\begin{multilinecode}
Example 3:
Input: [1, 1, 0, 1, 1, 0, 1, 0]
Output: [1, 1, 0, 1, 1, null, 1]
\end{multilinecode}
\includegraphics*[width=14cm]{figs/algo_814_3}
\end{example}

\textbf{Note}
\begin{itemize}
\item The binary tree will have at most \inlinecode{100 nodes}.
\item The value of each node will only be \inlinecode{0} or \inlinecode{1}.
\end{itemize}

\subsection*{Analysis}
This is a problem about the binary tree. However, it is not a complicated problem because we only need a preorder traversal and count the number of 1 in the subtree of a node. Then if the count is `0` of the subtree for a node, we just need to make the pointer to \inlinecode{null}.

We assume that the nodes number is $n$, then
\begin{itemize}
\item Time Complexity: $\mathcal{O}(n)$
\item Space Complexity: $\mathcal{O}(n)$ (We need to maintain a count for each node)
\end{itemize}

\subsection*{Solution}
\subsubsection*{C}
\begin{minted}[framesep=2mm,
baselinestretch=1.2,
bgcolor=codebackground,
fontsize=\footnotesize,
breaklines,
linenos]{c}
int count_one(struct TreeNode* r) {
    if (r == NULL) return 0;  // leaf node
    int left_count = count_one(r->left);
    int right_count = count_one(r->right);
    /**
    * I write this comment just to say I remember to free the memory,
    * but in this test, forget it.
    */
    if (left_count == 0) r->left = NULL;
    if (right_count == 0) r->right = NULL;
    return r->val + left_count + right_count;  // consider the value for this node itself.
}

struct TreeNode* pruneTree(struct TreeNode* root) {
    count_one(root);  // only need call this function
    return root;
}
\end{minted}

\newpage

