\tocless\section{476. Number Complement}
\label{algo:476}

\subsection*{Difficulty}
Easy

\subsection*{Tags}
Bitwise Operation

\subsection*{Description}
Given a positive integer, output its complement number. The complement strategy is to flip the bits of its binary representation.

\textbf{Note}
\begin{enumerate}
\item The given integer is guaranteed to fit within the range of a 32-bit signed integer.
\item You could assume no leading zero bit in the integer's binary representation.
\end{enumerate}

\begin{example}
\begin{multilinecode}
Input: 5
Output: 2

Explanation: The binary representation of 5 is 101 (no leading zero bits), and its complement is 010. So you need to output 2.
\end{multilinecode}
\end{example}

\begin{example}
\begin{multilinecode}
Input: 1
Output: 0

Explanation: The binary representation of 1 is 1 (no leading zero bits), and its complement is 0. So you need to output 0.
\end{multilinecode}
\end{example}

\subsection*{Analysis}
There are many method to solve this problem. For example, you can fetch each bit of this num and calculate its complementary bit. In our solution, we use a different way to calculate. The key point in this problem is how to ignore the leading zero of a number. For example, \inlinecode{~5 = 1...11010} and \inlinecode{~5 \& 7 = 2 = 010}. So, what we need to do is find the smallest $2^n - 1$ that larger than this number. Then we can calculate the complementary number without leading zeros by \inlinecode{~num \& $(2^n - 1)$}, where $n = \lfloor{\log_2{num}}\rfloor)$. Therefore, this problem transforms to how to fast calculate $\lfloor{\log_2{num}}\rfloor)$. In \href{http://www.graphics.stanford.edu/~seander/bithacks.html}{http://www.graphics.stanford.edu/~seander/bithacks.html}, we can find a way with $\mathcal{O}(\log(N))$ time complexity, in which $N$ is the bit length of this number.

\begin{itemize}
\item Time Complexity: $\mathcal{O}(\log(N))$
\item Space Complexity: $\mathcal{O}(1)$
\end{itemize}

\subsection*{Solution}
\subsubsection*{C}
\begin{minted}[framesep=2mm,
baselinestretch=1.2,
bgcolor=codebackground,
fontsize=\footnotesize,
breaklines,
linenos]{c}
int findComplement(int num) {
    if (num == 0) return 1;
    
    int v = num;
    int r;
    int shift;
    
    // calculate log2(num)
    r = (v > 0xFFFF) << 4; v >>= r;
    shift = (v > 0xFF) << 3; v >>= shift; r |= shift;
    shift = (v > 0xF) << 2; v >>= shift; r |= shift;
    shift = (v > 0x3) << 1; v >>= shift; r |= shift;
    r |= (v >> 1);
    
    return (~num) & ((2 << r) - 1);
}
\end{minted}

\newpage

