\section*{771. Jewels and Stones}
\label{algo:771}

\subsection*{Difficulty}
Easy

\subsection*{Tags}
Hash Table

\subsection*{Description}
You're given strings \inlinecode{J} representing the types of stones that are jewels, and \inlinecode{S} representing the stones you have. Each character in \inlinecode{S} is a type of stone you have. You want to know how many of the stones you have are also jewels.

The letters in \inlinecode{J} are guaranteed distinct, and all characters in \inlinecode{J} and \inlinecode{S} are letters. Letters are case sensitive, so \inlinecode{"a"} is considered a different type of stone from \inlinecode{"A"}.

\begin{example}
\begin{multilinecode}
Input: J = "aA", S = "aAAbbbb"
Output: 3
\end{multilinecode}
\end{example}

\begin{example}
\begin{multilinecode}
Input: J = "z", S = "ZZ"
Output: 0
\end{multilinecode}
\end{example}

\subsection*{Note}
\begin{itemize}
    \item \inlinecode{S} and \inlinecode{J} will consist of letters and have length at most 50.
    \item The characters in \inlinecode{J} are distinct.
\end{itemize}

\subsection*{Analysis}
This is an easy problem. All we need to do is to verify wether each letter in \inlinecode{S} exists in \inlinecode{J}.

Therefore, we can use a hash set to store \inlinecode{J}. Specifically, \inlinecode{J} is composed of letters (lower or upper) only, so we can use an array of \inlinecode{char} to store each letter in . After that, we only need to iterate \inlinecode{S} to calculate the number of jewels.

We assume the length of  is $m$, the length of \inlinecode{S} is $n$, then
\begin{itemize}
    \item Time complexity: $\mathcal{O}(m + n)$
    \item Space complexity: $\mathcal{O}(1)$ (256 ASCII chars)
\end{itemize}

\subsection*{Solution}
\subsubsection{C}
\begin{minted}[framesep=2mm,
baselinestretch=1.2,
bgcolor=codebackground,
fontsize=\footnotesize,
breaklines,
linenos]{c}
int numJewelsInStones(char* J, char* S) {
    char j_letters[256] = { 0 };  // initialize an array to store letters in J
    char c;
    for (int i = 0; (c = J[i]) != '\0'; ++i) {
        j_letters[c] = 1;  // set j_letters[c] = 1 means c appears in J
    }
    int jewel_number = 0;  // number of jewels
    for (int i = 0; (c = S[i]) != '\0'; ++i) {
        jewel_number += j_letters[c];  // if c appears in J, then we add a number
    }
    return jewel_number;
}
\end{minted}

\newpage

