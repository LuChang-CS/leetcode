\tocless\section{654. Maximum Binary Tree}
\label{algo:654}

\subsection*{Difficulty}
Medium

\subsection*{Tags}
Binary Tree

\subsection*{Description}
Given an integer array with no duplicates. A maximum tree building on this array is defined as follow:
\begin{enumerate}
	\item The root is the maximum number in the array.
	\item The left subtree is the maximum tree constructed from left part sub-array divided by the maximum number.
	\item The right subtree is the maximum tree constructed from right part sub-array divided by the maximum number.
\end{enumerate}

Construct the maximum tree by the given array and output the root node of this tree.

\begin{example}
\begin{multilinecode}
Input: [3, 2, 1, 6, 0, 5]
Output: return the tree root node representing the following tree:

      6
    /   \
   3     5
    \    / 
     2  0   
       \
        1
\end{multilinecode}
\end{example}

\textbf{Note}
\begin{itemize}
\item The size of the given array will be in the range [1, 1000].
\end{itemize}

\subsection*{Analysis}
At first sight, we can easily know that we can design a recursive algorithm to find the maximum value, take it as root, and do the same step on the left and right array. The worst time complexity is $\mathcal{O}(n^2)$ when array is in order, assuming that the number of array is $n$.

We can directly build this maximum binary tree in the following steps:
\begin{enumerate}
	\item Choose the first value as the root;
	\item For each successive value, if it is larger that the root, take it as root, and the previous root and its subtree as the left subtree of root (because previous root is the left array). Otherwise go to step 3;
	\item If the successive value is less than the root, then compare it with the right node of root (because this value is at the right array). If right node is null, take this value as right node. Otherwise go to step 2.
\end{enumerate}
\begin{itemize}
\item Time Complexity: Average is $\mathcal{O}(n{\log}{n})$ because we need to insert each value to a tree. Worst case is $\mathcal{O}(n^2)$, when original array is only descending but not ascending, which has less worst cases than recursive way. In this case, this algorithm degenerates to the insertion sort.
\item Space Complexity: $\mathcal{O}(1)$ (We don't need extra space).
\end{itemize}

\subsection*{Solution}
\subsubsection*{C}
\begin{minted}[framesep=2mm,
baselinestretch=1.2,
bgcolor=codebackground,
fontsize=\footnotesize,
breaklines,
linenos]{c}
struct TreeNode* constructMaximumBinaryTree(int* nums, int numsSize) {
    typedef struct TreeNode TreeNode;
    // we need a root pointer to the real root
    TreeNode *root = (TreeNode *)malloc(sizeof(TreeNode));
    root->right = NULL;
    for (int i = 0; i < numsSize; ++i) {
        int v = nums[i];
        TreeNode *p = root;
        // move to right and find a node's right child less than current value
        while (p->right != NULL && v < p->right->val) {
            p = p->right;
        }
        TreeNode *node = (TreeNode *)malloc(sizeof(TreeNode));
        node->val = v;
        node->left = p->right;  // null or less than current value should be node's left child
        node->right = NULL;
        p->right = node;  // replace previous right child
    }
    return root->right;  // return real root
}
\end{minted}

\newpage

