\tocless\section{682. Baseball Game}
\label{algo:682}

\subsection*{Difficulty}
Easy

\subsection*{Tags}
Stack \& Heap

\subsection*{Description}
You're now a baseball game point recorder.

Given a list of strings, each string can be one of the 4 following types:
\begin{enumerate}
\item \inlinecode{Integer} (one round's score): Directly represents the number of points you get in this round.
\item \inlinecode{"+"} (one round's score): Represents that the points you get in this round are the sum of the last two \inlinecode{valid} round's points.
\item \inlinecode{"D"} (one round's score): Represents that the points you get in this round are the doubled data of the last \inlinecode{valid} round's points.
\item \inlinecode{"C"} (an operation, which isn't a round's score): Represents the last \inlinecode{valid} round's points you get were invalid and should be removed.
\end{enumerate}

Each round's operation is permanent and could have an impact on the round before and the round after.

You need to return the sum of the points you could get in all the rounds.

\begin{example}
\begin{multilinecode}
Input: ["5","2","C","D","+"]
Output: 30
Explanation:
Round 1: You could get 5 points. The sum is: 5.
Round 2: You could get 2 points. The sum is: 7.
Operation 1: The round 2's data was invalid. The sum is: 5.  
Round 3: You could get 10 points (the round 2's data has been removed). The sum is: 15.
Round 4: You could get 5 + 10 = 15 points. The sum is: 30.
\end{multilinecode}
\end{example}

\begin{example}
\begin{multilinecode}
Input: ["5","-2","4","C","D","9","+","+"]
Output: 27
Explanation:
Round 1: You could get 5 points. The sum is: 5.
Round 2: You could get -2 points. The sum is: 3.
Round 3: You could get 4 points. The sum is: 7.
Operation 1: The round 3's data is invalid. The sum is: 3.  
Round 4: You could get -4 points (the round 3's data has been removed). The sum is: -1.
Round 5: You could get 9 points. The sum is: 8.
Round 6: You could get -4 + 9 = 5 points. The sum is 13.
Round 7: You could get 9 + 5 = 14 points. The sum is 27.
\end{multilinecode}
\end{example}

\textbf{Note:}
\begin{itemize}
\item The size of the input list will be between 1 and 1000.
\item Every integer represented in the list will be between -30000 and 30000.
\end{itemize}

\subsection*{Analysis}
Not too much to say, you just need to follow the instruction and use a stack to maintain all rounds.

We assume the total lengths of these string instructions are $n$, then
\begin{itemize}
\item Time Complexity: $\mathcal{O}(n)$
\item Space Complexity: $\mathcal{O}(n)$
\end{itemize}

\subsection*{Solution}
\subsubsection*{C}
\begin{minted}[framesep=2mm,
baselinestretch=1.2,
bgcolor=codebackground,
fontsize=\footnotesize,
breaklines,
linenos]{c}
int calPoints(char** ops, int opsSize) {
    int round[1000];
    int top = -1;
    int sum = 0;
    for (int i = 0; i < opsSize; ++i) {
        char *operation = ops[i];
        char op = operation[0];
        if ((op >= '0' && op <= '9') || (op == '-')) {
            // string integer to int
            int score = atoi(operation);
            sum += score;
            round[++top] = score;
        } else if (op == '+') {
            // top 2 elements in the stack
            int score = round[top] + round[top - 1];
            sum += score;
            round[++top] = score;
        } else if (op == 'D') {
            // double
            int score = round[top] << 1;
            sum += score;
            round[++top] = score;
        } else {
            // pop
            int score = -round[top--];
            sum += score;
        }
    }
    return sum;
}
\end{minted}

\newpage

