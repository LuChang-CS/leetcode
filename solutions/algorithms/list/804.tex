\tocless\section{804. Unique Morse Code Words}
\label{algo:804}

\subsection*{Difficulty}
Easy

\subsection*{Tags}
Hash Table

\subsection*{Description}
International Morse Code defines a standard encoding where each letter is mapped to a series of dots and dashes, as follows: \inlinecode{"a"} maps to \inlinecode{".-"}, \inlinecode{"b"} maps to \inlinecode{"-..."}, \inlinecode{"c"} maps to \inlinecode{"-.-."}, and so on.

For convenience, the full table for the 26 letters of the English alphabet is given below:
\begin{multilinecode}
[".-", "-...", "-.-.", "-..", ".", "..-.", "--.", "....", "..", ".---", "-.-", ".-..", "--", "-.", "---", ".--.", "--.-", ".-.", "...", "-", "..-", "...-", ".--", "-..-", "-.--", "--.."]
\end{multilinecode}

Now, given a list of words, each word can be written as a concatenation of the Morse code of each letter. For example, \inlinecode{"cab"} can be written as \inlinecode{"-.-.-....-"}, (which is the concatenation \inlinecode{"-.-."} + \inlinecode{"-..."} + \inlinecode{".-"}). We'll call such a concatenation, the transformation of a word.

Return the number of different transformations among all words we have.

\begin{example}
\begin{multilinecode}
Input: words = ["gin", "zen", "gig", "msg"]
Output: 2
Explanation:
The transformation of each word is:
"gin" -> "--...-."
"zen" -> "--...-."
"gig" -> "--...--."
"msg" -> "--...--."

There are 2 different transformations, "--...-." and "--...--.".
\end{multilinecode}
\end{example}

\subsection*{Note}
\begin{itemize}
    \item The length of \inlinecode{words} will be at most \inlinecode{100}.
    \item Each \inlinecode{words[i]} will have length in range \inlinecode{[1, 12]}.
    \item \inlinecode{words[i]} will only consist of lowercase letters.
\end{itemize}

\subsection*{Analysis}
This problem is the combination of Morse Code and a hash store. In this problem, we need to check repetition count of the Morse Code for each word in `words`. Therefore we can simply use a hash set to store appeared Morse Codes and check the existence of next.

We assume the size of \inlinecode{words} is $m$, and we have already known that each \inlinecode{words[i]} will have length in range \inlinecode{[1, 12]}, we assume the max length of a word is $n$, then
\begin{itemize}
\item Time complexity: $\mathcal{O}(mn)$ (We need to iterate all $m$ word in \inlinecode{words} and calculate the hash of this word, while checking the existence is $\mathcal{O}(1)$)
\end{itemize}

\subsection*{Solution}
\subsubsection*{C++}
\begin{minted}[framesep=2mm,
baselinestretch=1.2,
bgcolor=codebackground,
fontsize=\footnotesize,
breaklines,
linenos]{c++}
int uniqueMorseRepresentations(vector<string>& words) {
    string morse_table[] = { ".-", "-...", "-.-.", "-..", ".", "..-.", "--.", "....", "..", ".---", "-.-", ".-..", "--", "-.", "---", ".--.", "--.-", ".-.", "...", "-",  "..-", "...-", ".--", "-..-", "-.--", "--.." };
    unordered_set<string> morse_codes(100);  // the max size of words is 100
    for (vector<string>::iterator iter = words.begin(); iter != words.end(); ++iter) {
        string word = *iter;
        string code;
        code.reserve(50);
        for (string::iterator ch_iter = word.begin(); ch_iter != word.end(); ++ch_iter) {
            code += morse_table[*ch_iter - 'a'];  // calculate the morse code of each letter
        }
        morse_codes.insert(code);
    }
    return morse_codes.size();
}
\end{minted}

\newpage

