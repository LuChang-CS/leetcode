\tocless\section{728. Self Dividing Numbers}
\label{algo:728}

\subsection*{Difficulty}
Easy

\subsection*{Tags}
Math

\subsection*{Description}
A self-dividing number is a number that is divisible by every digit it contains.

For example, 128 is a self-dividing number because \inlinecode{128 \% 1 == 0}, \inlinecode{128 \% 2 == 0}, and \inlinecode{128 \% 8 == 0`}.

Also, a self-dividing number is not allowed to contain the digit zero.

Given a lower and upper number bound, output a list of every possible self dividing number, including the bounds if possible.

\begin{example}
\begin{multilinecode}
Input: left = 1, right = 22
Output: [1, 2, 3, 4, 5, 6, 7, 8, 9, 11, 12, 15, 22]
\end{multilinecode}
\end{example}

\textbf{Note}
\begin{itemize}
    \item The boundaries of each input argument are \inlinecode{1 <= left <= right <= 10000}.
\end{itemize}

\subsection*{Analysis}
This key problem is to fetch each digit of an integer and check whether this integer can divide each digit. In addition, as long as this integer contains \inlinecode{0}, it is not a self-dividing number.

\begin{itemize}
    \item Time complexity: $\mathcal{O}(right - left)$
    \item Space complexity: $\mathcal{O}(1)$
\end{itemize}

\subsection*{Solution}
\subsubsection{C}
\begin{minted}[framesep=2mm,
baselinestretch=1.2,
bgcolor=codebackground,
fontsize=\footnotesize,
breaklines,
linenos]{c}
int* selfDividingNumbers(int left, int right, int* returnSize) {
    int *results = (int *)malloc((right - left + 1) * sizeof(int));
    int count = 0;
    for (int i = left; i <= right; ++i) {
        int number = i;
        int flag = 1;
        while (number > 0) {
            // fetch each digits
            int digit = number % 10;
            // contains 0 or cannot be divided
            if (digit == 0 || i % digit != 0) {
                flag = 0;
                break;
            }
            // shift right in hex mode
            number /= 10;
        }
        if (flag == 1) {
            results[count++] = i;
        }
    }
    *returnSize = count;
    return results;
}
\end{minted}

\newpage

